%%% Experimental Results %%%

\subsection{Experimental Target and Considerations}

The Nuero-control algorithm presented is tested using a high-fidelity quadruped simulator, based on the
Open Dynamics Engine (ODE) \cite{OpenDynamicsEngine}, has been designed for testing control algorithms used on the 
BlueFoot quadruped platform. The simulated BlueFoot robot is outfitted with 16-revolute joints (4 per leg) 
which are controlled with independent proportional feedback loops.Controller gains have been chosen to match 
the response of servo actuators used on the physical system. Additionally, angular position and velocity feedback 
are available for each joint. The platform is also outfitted with a 12-axis inertial measurement unit (IMU) and 
binary-state contact sensors on each of its feet.
	\begin{figure}[t!]
		\centering
		\SetImage{\ImageWidthRatio}{gait_sequence.png}
		\caption{Trot gait sequence}
		\label{fig::gait_sequence}
		\ImagePostNoGap
	\end{figure}
To be consistent with BlueFoot's sensor outfit, the simulator does not emulate full state feedback. The
compensator is setup to perform as a reduced-state version of the originally present control scheme. In this reduced state 
representation, the trunk states $p_{b}$ and $\dot{p}_{b}$ are ignored. This does not change the general form of the controller 
at hand. Ideally, $\dot{p}_{b}$ would be estimated using a combination of vision-based odometry measurements and, ${p}_{b}$ 
would be estimated via a localization routine. 

Secondly, the angular position state of the robot's trunk is estimated via an Extended Kalman Filter emulation. This emulation
mixes the true trunk orientation signal, $\theta_{b}$, with a low-passed, Gaussian white noise signal to create the effect
of base-band drift.

Lastly, the force applied to each \Ith foot is estimated using a combination accelerometer and foot-contact data. 
Assuming a rigid system and uniform distribution, a rough estimate of the force 
applied to each \Ith planted foot, $\hat{f}_{i}$, can be generated by:
	\begin{equation}
		\hat{f}_{i} = \frac{\mu_{i}}{m_{T} \sum_{j=1}^{4}{\mu_{j}} } \left( \ddot{p}_{b} - \vec{g}\right)
	\end{equation}
where $m_{T}$ is a scalar representing the total system mass; $\mu_{i}\in \{0,1\}$ is the contact state of the \Ith
foot (a value $\mu_{i}=1$ represents contact); $\vec{g}$ is the gravity vector; and $\ddot{p}_{b}$ is the trunk
acceleration in the world frame. Ideally, these applied force vectors would be estimated using force-magnitude sensors 
placed  on each foot.

The following results will shown that the aforementioned approximations do not compromise the performance
of the compensator mechanism.

The compensator is applied to the simulated BlueFoot robot as it executes a CPG-driven trot gait depicted in \ref{fig::gait_sequence}.
This gaiting pattern is produced using a set of CPG parameters borrowed from \MissingRef.
During this gait, there are two planted feet, and two feet in flight at any time. The gait period, $T$, is adjusted to achieve 
some desired land-speed. This gait is the default used for mobilizing the BlueFoot platform.

The remaining experimental parameters are set as follows: mixing parameter is set to $\alpha=0.25$; learning-rate parameters are set to
$\beta=0.001$ and $\zeta=0.005$. The NARX-Network is setup with two hidden layers containing 50 neurons each. Each neuron is modeled
using a symmetric, $tanh(*)$, activation function. Output layer neurons are modeled using linear activation functions to avoid 
output-scaling saturation issues.


\subsection{Simulated System Results}


Figures \ref{fig::65mms_result} and \ref{fig::80mms_result} depict the roll and pitch trunk states during gaiting with alternating
Neuro-compensator activity. Is is shown in Figure \ref{fig::65mms_result} that during a moderately-paced gait used to achieve an
average land-speed of 65$\frac{mm}{s}$, the compensator (active between $t\in(10,20)$ and $t\in(30,40)$) decreases roll and pitch 
deviation from the zero-state by more than $50\%$. 

\begin{figure}[t!]
	\begin{subfigure}{0.5\textwidth}
		\centering
		\SetImage{\ImageWidthRatioSub}{regular_V_65mms_nN_50_nL_2_pos.png}
		\caption{ }
	\end{subfigure}
	\begin{subfigure}{0.5\textwidth}
		\centering
		\SetImage{\ImageWidthRatioSub}{regular_V_65mms_nN_50_nL_2_nns.png}
		\caption{ }
	\end{subfigure}
	\caption{ \textbf{a)} Trunk position \SSep \textbf{b)} Network MSE and Learning rate during gait for average ground speed 65$\frac{mm}{s}$  }
	\label{fig::65mms_result}
\end{figure}
\begin{figure}[t!]
	\begin{subfigure}{0.5\textwidth}
		\centering
		\SetImage{\ImageWidthRatioSub}{regular_V_80mms_nN_50_nL_2_pos.png}
		\caption{ }
	\end{subfigure}
	\begin{subfigure}{0.5\textwidth}
		\centering
		\SetImage{\ImageWidthRatioSub}{regular_V_80mms_nN_50_nL_2_nns.png}
		\caption{ }
	\end{subfigure}
	\caption{ \textbf{a)} Trunk position \SSep \textbf{b)} Network MSE and Learning rate during gait for average ground speed 80$\frac{mm}{s}$  }
	\label{fig::80mms_result}
\end{figure}