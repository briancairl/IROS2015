%%% System Dynamics

We begin by considering a general, free-floating, four legged robotic system with $m$ degrees of 
freedom per leg. This system is fully described by the state vector $\eta \RealVec{(6+4m)}$ and 
evolves as per the following dynamics:
	\begin{equation}
		M(\eta)\ddot{\eta} + C(\eta,\dot{\eta})\dot{\eta} + G(\eta) + \Delta{H} = \tau + J^T(\eta) f_{ext}
		\label{eq::normal_form_dynamics}
	\end{equation}
where $M(\eta)$, $C(\eta,\dot{\eta}$), $G(\eta)$ and $J(\eta)$ represent the system mass matrix, 
Coriolis matrix, gravity matrix and Jacobian, respectively \cite{Wieber2006}. $\Delta{H}$ 
has been included as a lump term for any dynamical uncertainties, such as friction or 
unmodeled coupling effects. Additionally, $f_{ext} \RealVec{6}$ represents the total external 
force-wrench applied to the system. 

The state vector, $\eta$, can be divided into the following sub components: 
$\eta = [ p_{b}^{T}, \theta_{b}^{T}, q^{T} ]^{T}$. $p_{b} \RealVec{3}$ and 
$\theta_{b} \RealVec{3}$ which represent the position and orientation, respectively, of the 
quadruped's trunk, in an arbitrarily placed world coordinate frame \cite{Wieber2015}.
$q \RealVec{4 m}$ is a vector of joint  variables, $m$ of which are contributed by each leg.

$\tau \RealVec{(6+4m)}$ represents a vector if generalized torque inputs and takes the form 
$\tau = [ 0_{1x6}, \tau_{q} ]^{T}$, where $\tau_{q}$ represents the torque inputs to the joints. 
It is important to note that the states we are most interested in controlling, $p_{b}$ and 
$\theta_{b}$, are not directly actuated, and must be controlled through a composite of leg motions.


\subsection{Dynamics in State-Space Form and an Approximate Discrete-Time Realization}


The dynamics in (\ref{eq::normal_form_dynamics}) can be realized in compact, state-space form by:
	\begin{equation}
		\begin{split}
		\dot{z}_{1} 				&= z_{2} \\
		\dot{z}_{2} 				&= M(z_{1})^{-1}(\tau + \Phi(z_{1},z_{2},f_{ext})) \\
		\Phi(z_{1},z_{2},f_{ext}) 	&= J^T(z_{1}) f_{ext} - C(z_{1},z_{2}) - G(z_{1}) - \Delta{H}
		\end{split}
		\label{eq::state_space_dynamics}
	\end{equation}
where $z_{1}=\eta$ and $z_{2}=\dot{\eta}$. The notation $\Phi(z_{1},z_{2},f_{ext})$ is invented
for convenience when dealing with $M(z_{1})$, $C(z_{1},z_{2})$, $G(z_{1})$, $J(z_{1})$ and $\Delta H$
as a composite dynamical term. This term will be referred to, simply, as $\Phi$ in the sections that follow.

Because the routine present in this paper is realized as a digital controller which, it is convenient to 
deal with the system dynamics, (\ref{eq::state_space_dynamics}), as a discrete-time approximation of 
the following form:
	\begin{equation}
		\begin{split}
		{z}_{1,k+1} &= {z}_{1,k} + ( {e}_{1,k}^{\Delta_{s}} + {z}_{2,k} )\Delta_{s} \\
		{z}_{2,k+1} &= {z}_{2,k} + M^{-1}_{1,k}( {e}_{2,k}^{\Delta_{s}} + \tau_{k} - \Phi_{k}) \Delta_{s} \\
		t 			&= t + \Delta_{s} k 
		\end{split}
		\label{eq::sampled_dynamics}
	\end{equation}
where $M_{1,k} = M(z_{1,k})$ and $\Delta_{s} \equiv (f_{s})^{-1}$, with $f_{s}$ defining a uniform 
state-sampling frequency in Hz. ${e}_{1,k}^{\Delta_{s}}$ and ${e}_{2,k}^{\Delta_{s}}$ are used to explicitly 
account for system discretization errors, which vary with with respect to the step-size, $\Delta_{s}$. These
discretization errors will be accounted using the learning mechanism, to be described later.


\subsection{Joint-Controller Dynamics}


An important part of the control algorithm at hand is its extensibility to systems that have been implemented 
with decentralized joint controllers. In the case where a decentralized joint control architecture is used, 
control over the torques applied at each is typically not explicit. To deal with this an inverse joint control 
model (or some adequate approximation) is needed for proper interfacing with the model-based controller.

In many cases, the controller dynamics of a revolute joint-position controllers can be described as a simple 
proportional (P), or P.D. control schemes. The BlueFoot platform, for example, utilizes 16 smart-servo controllers
that can be described via a P-control scheme. Using the introduced notations, BlueFoot's joint control scheme takes 
the follow aggregate form:
	\begin{equation}
		\tau = K_{s} (z_{1}^{r}-z_{1}) = 
		\left[\begin{array}{cc}
		0_{6\times6}	& 	0	\\
		0				&	k_{s} I_{(4m)\times(4m)}	\\
		\end{array}\right] (z_{1}^{r}-z_{1})
		\label{eq::servo_control_dynamics}
	\end{equation}
where $k_{s}>0$ is a constant, scalar gain. For the remainder of this paper, we will assume a decentralized 
joint-position control regime of this form. The control method at hand, however, is certainly not limited to 
platforms with this specific joint control architecture. With slight modifications, it is conceivable that this control
technique could be extended to any other decentralized joint control architecture with invertible or least-squares
invertible controller dynamics.